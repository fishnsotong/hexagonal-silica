\documentclass[a4paper,12pt]{article}

\usepackage[version=3]{mhchem}
\usepackage{amsmath,amsfonts,amsthm}
\usepackage{mathptmx}
\usepackage{xcolor}
\usepackage{titlesec}

%\titleformat{\section}[block]{\color{black}\Large\bfseries\filcenter}{}{1em}{}
\date{21  February 2018}
\author{Wayne Yeo Wei Zhong} % sets date blank and lists author as 

\title{Oral Defence Presentation Notes} % defines the title
\linespread{1.3}
\textwidth=16cm \oddsidemargin=0pt


   
	\begin{document}
	\maketitle
	
	\paragraph{1. Title slide.}Good afternoon members of the audience, I'm Wayne Yeo from Victoria Junior College. Today, I will be delivering a talk on my project: Building nanostructured porous silica materials directed by surfactants.
\bigbreak
I will begin with describing the background of the project, followed by a quick summary of our methods. I will then proceed to explain the results we have achieved, and where I plan to proceed from there.

	\paragraph{2. Interesting properties of silica nanostructures.}The first question I wish to address today, is why silica? Why did I choose this material to work with?
\bigbreak
Silica has a plethora of interesting properties that make it such a compelling material for use in nanostructures. Firstly, it is non-toxic and biocompatible. Secondly, it has displayed a wide flexibility of morphology. Thirdly, there is an immense potential for its surface functionalization and composition doping. 
\bigbreak
This makes it suitable for a wide range of \textit{in vivo} applications, ranging from drug encapsulation and delivery, to biomedical imaging.

	\paragraph{3. Templates allow us to direct silica nanostructure.}In order to achieve the functions that I've mentioned in my previous slide, there have been many efforts made to develop strategies which allow us to access different nanostructures of silica materials. Using a template allows us to create nanostructured silica with well-controlled size, shape and spatial arrangements.
\bigbreak
On the left panel are hollow silica nanoshells which are templated with hard spherical polystyrene templates. To access interesting, non-traditional morphologies such as these bowl-shaped capsules, and mesoporous silica, the templating of silica has expanded to the use of soft templates such as emulsions and surfactants. These different morphologies have found numerous applications, ranging from the petrochemical industry to drug delivery.
\bigbreak
In this particular study, we intend to investigate the use of soft templating to access novel silica nanostructures, in particular, surfactants as templates. Certain classes of surfactant, such as the double-tailed variety, have not been used as templating agents before, with most previous work focusing on single-tailed surfactants.

	\paragraph{4. Surfactants as a structure directing agent.}Surfactants have hydrophilic head groups as well as hydrophobic tail moieties. This causes them to self-assemble in aqueous solution. Furthermore, many surfactants have well-defined microstructures that varies as a function of solution temperature and surfactant concentration.
\bigbreak
The surfactant that I used as a structure directing agent is didodecyldimethylammonium phosphate. Shown on the screen is the major species of phosphate present in the solutions we use for synthesis. What is unique about this surfactant is that it has two hydrocarbon chains, and has not been used in previous studies involving silica synthesis. Ultimately, we want to explore and optimize its potential to direct silica nanostructure. 

\paragraph{5. Project scope and objectives.}Here are the three phases of my project. Firstly, I began by preparing the DDA surfactant used as a structure directing agent. Next, I proceeded to synthesize silica using a sol-gel synthesis process through a range of reaction conditions. In this project, I shall study the effect of systematically modifying surfactant concentration and solution temperature on the synthesis of templated silica. I hypothesise that we will observe consistent changes in morphological features in the nanosilica that can be attributed to changes in the surfactant system. this will be characterized with scanning electron microscopy (SEM).

\paragraph{6. Preparation of structure directing agent.}The DDA phosphate surfactant I intended to use is not commercially available, therefore, to make it in the lab I did an ion exchange of the surfactant's bromide counterion to a hydroxide counterion from a commercially available source of DDAB.
\bigbreak
After which, it was titrated with phosphoric acid till an appropriate equivalence point. During the titration, I used a pH meter to track the reaction and the successful formation of the phosphate species.


\paragraph{7. Preparation of structure directing agent (titration curve).}This is a typical titration curve that would be obtained, the desired endpoint was near pH 7, which corresponds to an ideal buffer region for the sol-gel synthesis of the silica.

\paragraph{8. Synthesis of silca nanostructures.}The silane precursor I'm using, tetramethoxysilane (TMOS), will then be added into a reaction vessel containing the surfactant we've prepared in the previous step. I then mix the reaction vessel vigorously using a vortex mixer, before waiting for 24 h for the sol-gel reaction to take place.
\bigbreak
The reaction temperatures I chose range from 0, 4, 8, 12, 16 and 20 $^\circ$C, and the surfactant concentrations ranged from 0.5, 1, 2, 3 and 5 wt. \%. I chose reaction conditions that are feasible, because beyond a certain concentration, the solution is too viscous for proper mixing to take place. 
\bigbreak
Previous work using the above scheme has demonstrated success, and we expect to be able to access different morphologies and dimensions of silica through conducting experiments along this matrix.


\paragraph{9. Silica morphologies at 0 $^\circ$C.}The next set of results were obtained, were done at 0 $^\circ$C on ice. 
\bigbreak
As we can see, at 0.5 wt.\% I see small silica nanobeads. With a 1.0 wt.\% surfactant concentration, a mixture of morphologies can be observed, indicating a transition. Small beads of nanosilica can be found alongside large polygonal plates with hexagonal dimensions.
\bigbreak
At 2.0 wt.\%, we observe that the nanobeads are nowhere to be found, only polygonal plates remain.
\bigbreak
This clearly defined transition of morphology from beads to plates demonstrates that we can alter the morphologies of silica nanostructures by varying the concentration of surfactant in solution.

\paragraph{10. Silica morphologies as a function of temperature and concentration.}As we expand the range of synthetic conditions the reaction was conducted under, we get the following results. The vertical axis represents an increase in concentration, and the horizontal axis represents an increase in temperature.
\bigbreak
As I've mentioned in my previous slide, as the concentration was increased from 0.5 wt.\% to 2.0 wt.\%, we see a general trend that the dimensions of the particles formed increases, and we can clearly observe a transition in morphology from nanobeads to polygonal plates.
\bigbreak
When I increase the surfactant concentration further at low temperatures to 5.0 wt.\%, I was able to observe another transition in morphology from hexagonal plates to concave toroidal structures. These concave structures are significantly smaller than the polygonal plates we've seen at lower concentrations, by a magnitude of half.
\bigbreak
As we increase the reaction temperature at low concentrations, we see a loss in templating ability by the surfactant. The wavy streaks we get resemble colloidal silica with no morphology or traces of surfactant residue we weren't able to remove by centrifugation. 
\bigbreak
With an increase in temperature, the reactions at high surfactant concentrations produced a range of intermediate morphologies ranging from spheroids, to donuts, and larger nanorod structures. This occurs before the eventual transition into spheroids, where we see a gradual shift towards more consistent morphologies and size distributions at higher temperatures and concentrations. Here we observe two main morphologies, smaller nanobeads and larger spheroids, some of which have a concavity.

\paragraph{11. Conclusions.}The results I present today demonstrate the ability that varying easily accessible reaction conditions, such as solution temperature and surfactant concentration will indeed have an impact on the templating behavior of the silica.
\bigbreak
This allows us to access different morphologies and dimensions of nanostructures, as seen in these two examples of polygonal plates and concave particles, within the same synthesis scheme.

\paragraph{12. Future work.}Now that we've studied the effects of varying temperature and concentration, there is an avenue to add another dimension into the matrix of reaction conditions, by systematically varying the pH of the reaction.
\bigbreak
Furthermore, these structures can be additionally modified through doping or surface functionalization, to grant them interesting properties. We can additionally explore the scale-up of this process, to assess its potential for industrial use.


\paragraph{13. Acknowledgements.}I would like to thank Victoria Junior College and the Institute of Materials Research and Engineering for allowing me to conduct this work, and my supervisors and mentors for their guidance and support.



		
	 
\end{document}