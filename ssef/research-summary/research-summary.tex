\documentclass[a4paper,12pt]{article}

\usepackage[version=3]{mhchem}
\usepackage{amsmath,amsfonts,amsthm}
\usepackage{mathptmx}
\usepackage{xcolor}
\usepackage{titlesec}

%\titleformat{\section}[block]{\color{black}\Large\bfseries\filcenter}{}{1em}{}
\date{}
\author{Wayne Yeo Wei Zhong} % sets date blank and lists author as 

\title{Research Summary} % defines the title
\linespread{1.3}
\textwidth=16cm \oddsidemargin=0pt


   
	\begin{document}
	\maketitle
	
	The employment of silica for use in biomedical applications, from drug encapsulation to biomedical imaging, has become increasingly widespread in recent decades. As such, there has been intense research on the processes used to synthesise nanosilica. Since the initial St\"ober process invented in 1968, there have been numerous advances that brought the synthesis of silica nanoparticles to aqueous and ambient conditions for convenient and eco-friendly processing.
	\bigbreak
	In addition, many efforts have been made to exert control over silica morphology through templating with a variety of materials. These strategies range from hard colloidal templates such as polystyrene nanospheres, to soft templates such as surfactants and biomolecules which produce silica nanostructures with various non-traditional morphologies that are functional in nanomedicine and the petrochemical industry, with surfactants displaying potential as a flexible templating agent due to the richness in their aggregation behaviour and variety of available self-assembled microstructures.
	\bigbreak
	Double-tailed surfactants such as didodecyldimethylammonium (\ce{DDA^+}) systems have been extensively studied in the literature, however, they remain underexplored as a soft templating agent for silica synthesis. Of particular interest is the creation of a robust and flexible fabrication approach that allows for the synthesis of a broad range of silica nanostructures, over the variation of a broad range of precise reaction conditions. Therefore, the purpose of this research is to study the effects of morphology of silica nanostructures templated by \ce{DDA^+} phosphate (\ce{DDAH_2PO_4}) surfactant when varying easily accessible reaction conditions \textit{in situ} such as surfactant concentration and reaction temperature. I hypothesise that we will observe consistent changes of morphological features in the nanostructured silica, where the variation in morphology can be attributed to changes in the surfactant system.
	\bigbreak
	The project is split into three phases. Firstly, I prepared the \ce{DDAH_2PO_4} templating agent from commercially available didodecyldimethylammonium bromide (DDAB) by conducting an ion-exchange and a titration. Secondly, a suitable matrix of experimental conditions at different surfactant concentrations from 0.5, 1.0, 2.0, 3.0 and 5.0 wt. \% and solution temperatures from 0, 4, 8, 12, 16 to 20 $^\circ$C were designed. The silica synthesis where silica was obtained from tetramethoxysilane (TMOS) through a sol-gel process was conducted at each of the stipulated reaction conditions. Thirdly, scanning electron microscopy (SEM) images of the nanostructures were obtained from a licensed operator, and we could observe qualitative relations between the changes in experimental conditions and variations in morphology.
	\bigbreak
	A number of trends have been identified from the data collected, and are described below. At low temperatures, an increase in surfactant concentration will result in clearly defined morphological transitions from nanobeads, to polygonal plates and concave toroidal particles. Across the range of concentrations that have been investigated in this study, an increase in temperature results in a gradual loss in templating ability by the surfactant, yielding colloidal silica at low surfactant concentrations and spheroidal nanoparticles at high surfactant concentrations.
	\bigbreak
	This research project provides a facile procedure for synthesising different morphologies of silica within the same synthesis scheme by varying easily accessible reaction parameters in a robust and flexible fashion. We have concluded that other factors, such as solution pH and doping small quantities of cations into solution will affect the surfactant self-assembly behaviour, possibly establishing new dimensions of control over silica nanostructure. The results have significant implications for novel bioengineered solutions, such as the creation of synthetic platelets. Conjugating conditionally activated thrombin on the silica discs can potentially allow for the mimicry of natural platelet functions, which will be an exciting and much-needed supplement to the current situation.
		
	 
	\end{document}