\documentclass[a4paper,12pt]{article}

\usepackage[version=3]{mhchem}
\usepackage{amsmath,amsfonts,amsthm}
\usepackage{mathptmx}
\usepackage{xcolor}
\usepackage{titlesec}
\pagenumbering{gobble}

\titleformat{\section}[block]{\color{black}\Large\bfseries\filcenter}{}{1em}{}
\date{}
\title{Abstract} % defines the title
\linespread{1.3}
\textwidth=16cm \oddsidemargin=0pt


   
	\begin{document}
	\section*{Abstract}
	
	\vspace{1.5\baselineskip} % Whitespace below the title
	
	
	 Silica nanomaterials have found use in biomedical applications due to their biocompatibility and non-toxicity. These utilizations are made possible largely due to the wide range of structures made available through templated approaches to its sol-gel synthesis. Templated synthesis enables us to direct the formation of silica nanostructures into forms that are otherwise difficult to obtain, allowing the exertion of a fine degree of synthetic control over the morphology, dimensions and architecture of the nanostructures. In previous work, single-tailed surfactants have been used as soft templates to produce mesoporous silica materials. This study investigates the use of double-tailed didodecyldimethylammonium (\ce{DDA^+}) phosphate surfactants as a structure-directing agent for the soft template sol-gel synthesis of silica at ambient conditions in aqueous solution. The effects of varying accessible reaction parameters such as surfactant concentration and solution temperature on resulting silica morphology was characterised. In this research, we have demonstrated that varying these reaction conditions will indeed have an impact on the templating behaviour of the surfactant, resulting in morphological transitions from nanobeads to polygonal plates and toroidal concave particles with an increase in surfactant concentration, as well as a gradual loss in templating ability with an increase in solution temperature. This allows us to access different morphologies and dimensions of nanostructures within the same synthesis scheme in a robust and flexible fashion. Future studies to establish new dimensions of control over silica nanostructure can be initiated by varying the solution pH and doping small quantities of electrolyte into solution.
	 
	 
	\end{document}