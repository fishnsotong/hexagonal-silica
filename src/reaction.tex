\documentclass[10pt,twoside,a4paper]{article}

\usepackage[version=3]{mhchem}
\usepackage{amsmath,amsfonts,amsthm}

\newtheorem{thm}{Derivation}

\begin{document}
\section*{Supplementary Information}
     
    \begin{thm}The expression of the fractional composition of various counterions as a function of hydronium ion concentration.
    \end{thm}


	\begin{proof} First, we consider the acidity dissociation $K_a$ values of each successive deprotonation of \ce{H_3PO_4}.
	
		\begin{equation*}
 		\ce{SiO(CH3)4(l) + 4H2O(l) -> Si(OH)4(aq) + 4HOCH3(aq)}
 		\end{equation*}
 		\begin{equation*}
 		\ce{Si(OH)4(aq) + OH^-(aq) -> Si(OH)3O^-(aq) + H2O(l)}
 		\end{equation*}
 		\begin{equation*}
 		\ce{Si(OH)4(aq) + Si(OH)3O^-(aq) -> (HO)3Si-O-Si(OH)3(s) + OH^-(aq)}
 		\end{equation*}
 		
 	
 		
 	Thereafter, we consider that the initial concentration of phosphoric acid in aqueous solution $P$, can be expressed as
 	the sum of the concentration of phosphoric acid in all its forms.
 		
 		\begin{eqnarray}
 		\text{initial concentration } P & = & \ce{[H3PO4]}_\text{as prepared} \nonumber \\ 
 		& = & \ce{[H3PO4] + [H2PO4^-] + [HPO4^{2-}] + [PO4^{3-}]} \nonumber 		
 		\end{eqnarray}
 		
 	We then manipulate the $K_a$ expressions such that all species concentrations can be expressed in terms of \ce{[H_3PO_4]} and \ce{[H3O+]}, along with the thr

 		
	\end{proof}
	

\end{document}